\documentclass{article}
\usepackage{mypkg}

\begin{document}
{\Huge \center{Notes on Baby Rudin}}

\section{Continuity}
The first thing we look at is preservation of connectedness under a
continuous function: this is what allows the intermediate value theorem
to be true.
\begin{thm}
  Let $f: X \mapsto Y$ be a continuous function between metric spaces $X$
  and $Y$. Let $E \subset X$ be connected. Then $f(E)$ is connected.
\end{thm}
\begin{proof}
  We prove the contrapositive--- if we have non-empty, \emph{separated} sets
  $A \subset f(E)$ and $B \subset f(E)$,
  their preimages within $E$ are separated. Then if
  \(f(E) = A \cup B\) where $A$ and $B$ are separated, $E$ must be a
  non-connected set, which completes our proof.

  Assume, then, that $A$, $B$ are subsets of $f(E)$ and
  $\bar{A}\cap B = \phi = A \cap \bar{B}$. Define the sets
  \begin{eqnarray*}
    G & = f^{-1}(A) \cap E \\
    H & = f^{-1}(B) \cap E
  \end{eqnarray*}
  Then we have $G$ and $H$ are disjoint (else $A$ and $B$ would not be
  disjoint). We also have $G$ and $H$ nonempty, else $A$ and/or $B$
  would be empty.

  To show that $G$ and $H$ are separated, without loss of generality we
  will only show that $\bar G \cup H = \phi$. Assume $g$ is a limit
  point of $G$. If we cannot find such a point, we
  are done, so assume one exists. As $g$ is a limit point, there exists
  a sequence of points $\{g_i\} \subset G$ such that
  $g_i \rightarrow g$. But then by continuity,
  $f(g_i) \rightarrow f(g)$, or in other words, $f(g)$ is a limit point
  of $f(A)$ (as each $f(g_i) \in A$).

  By the separation of $A$ and $B$, we have $f(g)\not\in B$. Thus
  $g\not\in H$, otherwise we would have $f(g)\in B$. Therefore,
  $\overline G \cup H = \phi$, and we are done.
 
\end{proof}
\begin{note*}
  Rudin has a slightly more elegant, but perhaps less satisfying
  proof. Let $A$, $B$, $G$, $H$ be as above. As $f$ is continuous, we
  have $\inv{f}(\bar A)$ closed. Thus
  \[G \subset f^{-1}(\bar A) \,\implies\, \bar G \subset \inv{f}(\bar A) \]
  As $\bar A \cap B = \phi$, we must have $\bar G \cap H = \phi$.
\end{note*}
\end{document}
