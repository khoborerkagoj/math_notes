\documentclass{article}
\usepackage{mypkg}

\begin{document}
{\center{\huge Number theory notes}}

\section{Divisibility and prime factorization}
A lot of the following discussion is inspired by
Theorem~\ref{composite_2_power}, which is from Underwood Dudley,
Section~2, problem~14.

\begin{thm}\label{composite_2_power}
  If $n\in \nat$ is composite, so is $2^n - 1$.
\end{thm}
To prove this, we prove:
\begin{prop}\label{2p_2pq}
  If $n = pq$, then $(2^p - 1)\mid (2^{pq}-1)$
\end{prop}
\begin{proof}
  \[x^q - 1 = (x - 1)(1 + x + x^2 +\cdots+ x^{q-1})\]
  From which the result follows by setting $x = 2^p$.
\end{proof}

This now immediately proves Theorem \ref{composite_2_power} by setting
$n = pq$, which implies $(2^p - 1) \mid (2^n - 1)$. Since $p > 1$ (as $n$
is composite), $2^p - 1 > 1$ which means $2^n - 1$ is composite.

\begin{cor}\label{2a_2b_gcd_match}
  If $(a,b) > 1$, then $(2^a - 1, 2^b - 1) > 1$.
\end{cor}
\begin{proof}
  Similar to the previous theorem. Note that the proof of
  proposition~\ref{2p_2pq} does not depend on the primality of $p$ or
  $q$. Thus if $(a,b) = g$ and $a = g\alpha$, $b = g\beta$, then we have
  \begin{alignat*}{2}
    (2^g - 1) & \mid (2^{g\alpha} - 1) &= 2^a - 1\\
    (2^g - 1) & \mid (2^{g\beta}  - 1) &= 2^b - 1
  \end{alignat*}
  And $(2^g - 1) > 1)$ since $g > 1$ is a common divisor.
\end{proof}

\begin{discussion}
It appears that when (not verified) $p$ and $q$ are
relatively prime, in other words, when $(p,q) = 1$, and $a = 2^p$, we have
\[(2^q - 1) \mid (1 + a + a^2 + a^3 + \cdots + a^{q-1})\]

But it does not hold when $a$ is not of that form. For instance, when
$q = 3$ and $a = 3$ (note, not $2^3$), we have $1 + 3 + 3^2 = 13$ which
is not divisible by $2^3 - 1 = 7$. But, when $a = 2^2 = 4$, we have
\[7 \mid (1 + 4 + 4^2) = 21\]

When $a = 2^3 = 8$, we get $1 + 8 + 8^2 = 73$, which is prime! But in
this case $p = q$.

Does this hold in general?

An outline of proof would require us to show the converse of Corollary
\ref{2a_2b_gcd_match}, i.e.
$(a,b) = 1 \implies (2^a - 1,2^b-1) = 1$.
This is numerically seen to be true for values up to $a,b=100$, but no
proof. If we assume this to be true, we can say (for $a = 2^p$)
\[ S = (1 + a + a^2 + a^3 + \cdots + a^{q-1}) = \frac{2^{pq}-1}{2^p-1}\]
Since $2^{pq - 1}$ is divisible by both $2^p - 1$ and $2^q - 1$,
\[(2^p - 1,2^q-1) = 1 \implies (2^q - 1) | S\].

\end{discussion}
\end{document}
