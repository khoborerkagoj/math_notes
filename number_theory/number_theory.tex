\documentclass{article}
\usepackage{amsmath}
\usepackage{amsthm}
\usepackage{amssymb}

\newtheorem{thm}{Theorem}
\newtheorem{prop}{Proposition}

\newcommand{\nat}{\ensuremath{\mathbb{N}}}

\begin{document}
{\centering{\huge Number theory notes}}

\section{Underwood Dudley}
\begin{thm}\label{composite_2_power}
  If $n\in \nat$ is composite, so is $2^n - 1$.
\end{thm}
To prove this, we prove:
\begin{prop}
  If $n = pq$, then $(2^p - 1)\mid (2^{pq}-1)$
\end{prop}
\begin{proof}
  \[x^q - 1 = (x - 1)(1 + x + x^2 +\cdots+ x^{q-1})\]
  From which the result follows by setting $x = 2^p$.
\end{proof}

This now immediately proves Theorem \ref{composite_2_power} by setting
$n = pq$, which implies $(2^p - 1) \mid (2^n - 1)$. Since $p > 1$ (as $n$
is composite), $2^p - 1 > 1$ which means $2^n - 1$ is composite.

Note this also means that
\[(2^q - 1) \mid\sum_{j=0}^{q-1}2^{j\cdotp}\]
\end{document}
